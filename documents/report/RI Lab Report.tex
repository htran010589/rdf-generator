\documentclass{acm_proc_article-sp}

\usepackage{pdfpages}
\usepackage{cite}
\usepackage{algorithm}
\usepackage{algorithmic}

\begin{document}

\title{Projection of Binary Predicates}
\numberofauthors{1}
\author{
\alignauthor
Student: Hai Dang Tran\\Matriculation Number: 2557779
}
\date{09 January 2016}

\maketitle

\section{INTRODUCTION}
This report proposes a simple predicate projection algorithm to mine interesting rules from RDF\footnote{https://en.wikipedia.org/wiki/Resource_Description_Framework} graph. To be specific, the input of the algorithm is a matrix $A \in \mathbb{R}^{n \times m}$ in which $n$ can be very large and $A$ can be represented as a stream of row vectors $A_{i}$. The algorithm tends to find a matrix $B \in \mathbb{R}^{l \times m}$ with $l \ll n$ and $B^TB$ well approximates $A^TA$, that is, $B^TB \approx A^TA$, more formally:

$\|A^TA - B^TB\| \leq 2\|A\|^2_{f} / l$

However, in case $\|A\|_{f}$ is large, this bound can be very loose. This problem will be discussed in details in the last section.

The order of the content in this summary is as follows. Firstly, literature review is introduced. Secondly, the main algorithm of the publication~\cite{ref1} is discussed in details. Finally, we talk about experimental results and some critical questions about this publication.\\

\section{Projection Strategy}

Matrix sketching is a problem of finding a new matrix which is much smaller than the original one. Besides, with the same operation executed on the new or original matrices, the results are not much different~\cite{ref1}. Matrix sketching algorithms pay attention to streaming context where the each row vector of the original matrix is accessed at most one time.

\begin{algorithm}
\caption{Frequent Directions~\cite{ref1}}
\label{algo1}
\begin{algorithmic}
\REQUIRE $l, A \in \mathbb{R}^{n \times m}$
\STATE $B \leftarrow$ zero matrix $\in \mathbb{R}^{l \times m}$
\FOR{$i \in [n]$}
\IF{$B$ has no zero row vectors}
\STATE $[U, \Sigma, V] \leftarrow$ SVD$(B)$
\STATE $\delta \leftarrow \sigma^2_{l/2}$
\STATE $\hat{\Sigma} \leftarrow \sqrt{max(\Sigma^2 - I_{l} \delta, 0)}$
\STATE $B \leftarrow \hat{\Sigma} V^T$
\ENDIF
\STATE Insert $A_{i}$ into a zero row vector of $B$
\ENDFOR
\end{algorithmic}
\end{algorithm}

\section{Future Work}

Several approaches for matrix sketching problem are as follows. The first approach tends to find a sparse matrix from the original one. Because of the sparsity, some operations can be performed efficiently on this sketch~\cite{ref2, ref3, ref14}. In the second approach, the combination of some rows are calculated to form a new sketch matrix~\cite{ref4, ref5, ref15, ref16}. The \textit{random projection} method in the Experiments part is corresponding to above mentioned algorithms. Besides, another combination-based method with the name \textit{hashing}~\cite{ref6} will also be tested and compared with the main method. The third approach pays attention to choosing a subset of row or column vectors to create sketch matrix. Another name for this approach is Column Subset Selection Problem which attracts interest in~\cite{ref7, ref8, ref9, ref10, ref11}. In the streaming model of this approach, row sampling is conducted. Publications in~\cite{ref11, ref7, ref12, ref13, ref17} focus on bounds of sampling each row vector with the probability derived from its norm. This algorithm is also \textit{sampling} method in the Experiments part.


\bibliographystyle{abbrv}
\bibliography{sigproc}

\end{document}
